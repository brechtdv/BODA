\documentclass[]{book}
\usepackage{lmodern}
\usepackage{amssymb,amsmath}
\usepackage{ifxetex,ifluatex}
\usepackage{fixltx2e} % provides \textsubscript
\ifnum 0\ifxetex 1\fi\ifluatex 1\fi=0 % if pdftex
  \usepackage[T1]{fontenc}
  \usepackage[utf8]{inputenc}
\else % if luatex or xelatex
  \ifxetex
    \usepackage{mathspec}
  \else
    \usepackage{fontspec}
  \fi
  \defaultfontfeatures{Ligatures=TeX,Scale=MatchLowercase}
\fi
% use upquote if available, for straight quotes in verbatim environments
\IfFileExists{upquote.sty}{\usepackage{upquote}}{}
% use microtype if available
\IfFileExists{microtype.sty}{%
\usepackage[]{microtype}
\UseMicrotypeSet[protrusion]{basicmath} % disable protrusion for tt fonts
}{}
\PassOptionsToPackage{hyphens}{url} % url is loaded by hyperref
\usepackage[unicode=true]{hyperref}
\hypersetup{
            pdftitle={Burden of Disease Assessment: A Practical Guide},
            pdfauthor={Brecht Devleesschauwer},
            pdfborder={0 0 0},
            breaklinks=true}
\urlstyle{same}  % don't use monospace font for urls
\usepackage{natbib}
\bibliographystyle{apalike}
\usepackage{longtable,booktabs}
% Fix footnotes in tables (requires footnote package)
\IfFileExists{footnote.sty}{\usepackage{footnote}\makesavenoteenv{long table}}{}
\usepackage{graphicx,grffile}
\makeatletter
\def\maxwidth{\ifdim\Gin@nat@width>\linewidth\linewidth\else\Gin@nat@width\fi}
\def\maxheight{\ifdim\Gin@nat@height>\textheight\textheight\else\Gin@nat@height\fi}
\makeatother
% Scale images if necessary, so that they will not overflow the page
% margins by default, and it is still possible to overwrite the defaults
% using explicit options in \includegraphics[width, height, ...]{}
\setkeys{Gin}{width=\maxwidth,height=\maxheight,keepaspectratio}
\IfFileExists{parskip.sty}{%
\usepackage{parskip}
}{% else
\setlength{\parindent}{0pt}
\setlength{\parskip}{6pt plus 2pt minus 1pt}
}
\setlength{\emergencystretch}{3em}  % prevent overfull lines
\providecommand{\tightlist}{%
  \setlength{\itemsep}{0pt}\setlength{\parskip}{0pt}}
\setcounter{secnumdepth}{5}
% Redefines (sub)paragraphs to behave more like sections
\ifx\paragraph\undefined\else
\let\oldparagraph\paragraph
\renewcommand{\paragraph}[1]{\oldparagraph{#1}\mbox{}}
\fi
\ifx\subparagraph\undefined\else
\let\oldsubparagraph\subparagraph
\renewcommand{\subparagraph}[1]{\oldsubparagraph{#1}\mbox{}}
\fi

% set default figure placement to htbp
\makeatletter
\def\fps@figure{htbp}
\makeatother

\usepackage{booktabs}
\usepackage{amsthm}
\makeatletter
\def\thm@space@setup{%
  \thm@preskip=8pt plus 2pt minus 4pt
  \thm@postskip=\thm@preskip
}
\makeatother

\title{Burden of Disease Assessment: A Practical Guide}
\author{Brecht Devleesschauwer}
\date{2020-05-24}

\begin{document}
\maketitle

{
\setcounter{tocdepth}{1}
\tableofcontents
}
\chapter*{Preface}\label{preface}
\addcontentsline{toc}{chapter}{Preface}

Disability-Adjusted Life Years (DALYs) have become a key indicator in
descriptive epidemiology. DALYs represent the number of healthy life
years lost due to ill health and mortality, and allow comparing the
population health impact of diseases, injuries and risk factors.

Although the DALY concept has been introduced nearly 30 years ago, there
is still little guidance available on their calculation. This book aims
to address this gap, through a combination of theoretical sections,
simplified examples, and real-life experiences.

This book is the result of interactions and collaborations within the
European Burden of Disease Network (COST Action CA18218), supported by
COST (cooperation in science and technology). Further information on the
network is available via \url{https://www.burden-eu.net}.

\section*{Why read this book}\label{why-read-this-book}
\addcontentsline{toc}{section}{Why read this book}

This book is primarily intended for students, researchers and public
health professionals interested in learning how to calculate DALYs.

However, it should also be noted that the \textbf{best way of learning
is by doing}. We therefore hope that this book can encourage you to get
started with your own calculation examples.

\section*{Structure of the book}\label{structure-of-the-book}
\addcontentsline{toc}{section}{Structure of the book}

The first part of the book is dedicated to the basic concepts of DALY
calculations. Starting from simple examples, different layers of
complexity will be introduced.

The second part of the book is dedicated to national burden of disease
studies.

\chapter*{About the authors}\label{about-the-authors}
\addcontentsline{toc}{chapter}{About the authors}

\section*{Brecht Devleesschauwer}\label{brecht-devleesschauwer}
\addcontentsline{toc}{section}{Brecht Devleesschauwer}

Some info here.

\url{https://twitter.com/brechtdv}

\part{Calculating DALYs}\label{part-calculating-dalys}

\chapter{Introduction}\label{introduction}

The ultimate goal of public health policy is to protect and promote the
population's health (Devleesschauwer et al., 2014a). This requires
information on the health status of the population, often referred to as
the ``burden of disease''. In order to make relevant decisions and set
appropriate priorities, policy makers need to be informed about the size
of health problems in the population, the groups that are particularly
at risk, and the trends in the state of health over time. In addition,
an accurate estimate of the population's health status can be used for
determining the expected health care use and is vital for prioritizing
effective interventions and evaluating their impact and
cost-effectiveness (Tan-Torres Edejer et al., 2003).

As public health is a multifactorial phenomenon with many facets, the
disease burden of the population can be described by a variety of
indicators. Typical indicators of population health are life expectancy,
cause-specific mortality rates, numbers of new and existing cases of
specific diseases (i.e., incidence and prevalence), perceived health,
the occurrence of physical and mental limitations and disability, but
also more indirect measures, such as absenteeism, incapacity of work,
and the use of medical facilities and the associated costs. However, all
these indicators highlight only one facet of public health, i.e., either
mortality or morbidity.

Summarizing public health in terms of mortality-based indicators, such
as life expectancy, dates from the time when only reliable data for
mortality existed. In many countries, however, one has been confronted
with ageing populations and an epidemiological transition of public
health problems. The importance of early mortality due to plagues and
famines has been replaced by chronic, non-communicable diseases, while
communicable diseases remain a real threat, causing a ``double burden''
(Marshall, 2004). Cardiovascular diseases and cancers have replaced
infectious diseases as the main causes of death. However, these diseases
are also associated with an important morbidity component, due to the
life prolonging effect of continuously improving medical practice
(Jelenc et al., 2012). Moreover, not only an extended life expectancy
per se is aimed for, living these extra years in good health has become
just as important (Bryant et al., 2001). As a result, current health
policy requires a global overview of public health, one that combines
morbidity and mortality and takes account of health-related quality of
life (Robine et al. 2013).

Given the importance of combining morbidity and mortality, several
summary measures of population health (SMPH) have been proposed and
implemented (Murray et al., 2000; Table 1). SMPHs may be divided into
two broad families: health expectancies or experiences and health gaps,
but all have in common that they use ``time'' as the common measure for
quantifying health or health loss. The most powerful SMPHs are those
that are able to combine morbidity and mortality into a single figure.

\begin{longtable}[]{@{}lll@{}}
\caption{(\#smph) Classification of summary measures of population
health}\tabularnewline
\toprule
\begin{minipage}[b]{0.08\columnwidth}\raggedright\strut
\strut
\end{minipage} & \begin{minipage}[b]{0.50\columnwidth}\raggedright\strut
Health Experience\strut
\end{minipage} & \begin{minipage}[b]{0.34\columnwidth}\raggedright\strut
Health Gap\strut
\end{minipage}\tabularnewline
\midrule
\endfirsthead
\toprule
\begin{minipage}[b]{0.08\columnwidth}\raggedright\strut
\strut
\end{minipage} & \begin{minipage}[b]{0.50\columnwidth}\raggedright\strut
Health Experience\strut
\end{minipage} & \begin{minipage}[b]{0.34\columnwidth}\raggedright\strut
Health Gap\strut
\end{minipage}\tabularnewline
\midrule
\endhead
\begin{minipage}[t]{0.08\columnwidth}\raggedright\strut
Mortality\strut
\end{minipage} & \begin{minipage}[t]{0.50\columnwidth}\raggedright\strut
Life Expectancy\strut
\end{minipage} & \begin{minipage}[t]{0.34\columnwidth}\raggedright\strut
Potential Years of Life Lost(Years of Potential Life Lost)Standard
Expected Years of Life Lost\strut
\end{minipage}\tabularnewline
\begin{minipage}[t]{0.08\columnwidth}\raggedright\strut
Morbidity\strut
\end{minipage} & \begin{minipage}[t]{0.50\columnwidth}\raggedright\strut
Quality-Adjusted Life Year\strut
\end{minipage} & \begin{minipage}[t]{0.34\columnwidth}\raggedright\strut
Years Lived with Disability\strut
\end{minipage}\tabularnewline
\begin{minipage}[t]{0.08\columnwidth}\raggedright\strut
Morbidity \& Mortality\strut
\end{minipage} & \begin{minipage}[t]{0.50\columnwidth}\raggedright\strut
Active Life ExpectancyDisability-Free Life ExpectancyHealthy Life
YearsQuality-Adjusted Life ExpectancyDisability-Adjusted Life
Expectancy\strut
\end{minipage} & \begin{minipage}[t]{0.34\columnwidth}\raggedright\strut
Disability-Adjusted Life Year\strut
\end{minipage}\tabularnewline
\bottomrule
\end{longtable}

Driven by the influential Global Burden of Disease (GBD) projects
initiated in the early 1990s (Murray and Lopez, 1996), the
Disability-Adjusted Life Year (DALY) has become the dominant SMPH for
quantifying burden of disease. The DALY metric has therefore been
selected as key SMPH for the Belgian National Burden of Disease study.
DALYs measure the health gap from a life lived in perfect health, and
quantify this health gap as the number of healthy life years lost due to
morbidity and mortality. Although the basic DALY formulas are rather
straightforward, the calculation of DALYs, like any other SMPH, requires
several assumptions, some of which are not always obvious. Furthermore,
DALY-based burden of disease studies are almost always confronted by
uncertainties and almost always require manipulations of epidemiological
data.

\chapter{Basic concepts}\label{basic-concepts}

\chapter{Data needs}\label{data-needs}

\chapter{Disability weights}\label{disability-weights}

\chapter{Comorbidity}\label{comorbidity}

\chapter{Residual life expectancy}\label{residual-life-expectancy}

\chapter{Risk factors}\label{risk-factors}

\chapter{Quantifying uncertainty}\label{quantifying-uncertainty}

\bibliography{book.bib,packages.bib}

\end{document}
