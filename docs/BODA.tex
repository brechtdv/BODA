\documentclass[]{book}
\usepackage{lmodern}
\usepackage{amssymb,amsmath}
\usepackage{ifxetex,ifluatex}
\usepackage{fixltx2e} % provides \textsubscript
\ifnum 0\ifxetex 1\fi\ifluatex 1\fi=0 % if pdftex
  \usepackage[T1]{fontenc}
  \usepackage[utf8]{inputenc}
\else % if luatex or xelatex
  \ifxetex
    \usepackage{mathspec}
  \else
    \usepackage{fontspec}
  \fi
  \defaultfontfeatures{Ligatures=TeX,Scale=MatchLowercase}
\fi
% use upquote if available, for straight quotes in verbatim environments
\IfFileExists{upquote.sty}{\usepackage{upquote}}{}
% use microtype if available
\IfFileExists{microtype.sty}{%
\usepackage[]{microtype}
\UseMicrotypeSet[protrusion]{basicmath} % disable protrusion for tt fonts
}{}
\PassOptionsToPackage{hyphens}{url} % url is loaded by hyperref
\usepackage[unicode=true]{hyperref}
\hypersetup{
            pdftitle={Burden of Disease Assessment: A Practical Guide},
            pdfauthor={Brecht Devleesschauwer},
            pdfborder={0 0 0},
            breaklinks=true}
\urlstyle{same}  % don't use monospace font for urls
\usepackage{natbib}
\bibliographystyle{apalike}
\usepackage{longtable,booktabs}
% Fix footnotes in tables (requires footnote package)
\IfFileExists{footnote.sty}{\usepackage{footnote}\makesavenoteenv{long table}}{}
\usepackage{graphicx,grffile}
\makeatletter
\def\maxwidth{\ifdim\Gin@nat@width>\linewidth\linewidth\else\Gin@nat@width\fi}
\def\maxheight{\ifdim\Gin@nat@height>\textheight\textheight\else\Gin@nat@height\fi}
\makeatother
% Scale images if necessary, so that they will not overflow the page
% margins by default, and it is still possible to overwrite the defaults
% using explicit options in \includegraphics[width, height, ...]{}
\setkeys{Gin}{width=\maxwidth,height=\maxheight,keepaspectratio}
\IfFileExists{parskip.sty}{%
\usepackage{parskip}
}{% else
\setlength{\parindent}{0pt}
\setlength{\parskip}{6pt plus 2pt minus 1pt}
}
\setlength{\emergencystretch}{3em}  % prevent overfull lines
\providecommand{\tightlist}{%
  \setlength{\itemsep}{0pt}\setlength{\parskip}{0pt}}
\setcounter{secnumdepth}{5}
% Redefines (sub)paragraphs to behave more like sections
\ifx\paragraph\undefined\else
\let\oldparagraph\paragraph
\renewcommand{\paragraph}[1]{\oldparagraph{#1}\mbox{}}
\fi
\ifx\subparagraph\undefined\else
\let\oldsubparagraph\subparagraph
\renewcommand{\subparagraph}[1]{\oldsubparagraph{#1}\mbox{}}
\fi

% set default figure placement to htbp
\makeatletter
\def\fps@figure{htbp}
\makeatother

\usepackage{booktabs}
\usepackage{amsthm}
\makeatletter
\def\thm@space@setup{%
  \thm@preskip=8pt plus 2pt minus 4pt
  \thm@postskip=\thm@preskip
}
\makeatother

\title{Burden of Disease Assessment: A Practical Guide}
\author{Brecht Devleesschauwer}
\date{2020-05-24}

\begin{document}
\maketitle

{
\setcounter{tocdepth}{1}
\tableofcontents
}
\chapter*{Preface}\label{preface}
\addcontentsline{toc}{chapter}{Preface}

Disability-Adjusted Life Years (DALYs) have become a key indicator in
descriptive epidemiology. DALYs represent the number of healthy life
years lost due to ill health and mortality, and allow comparing the
population health impact of diseases, injuries and risk factors.

Although the DALY concept has been introduced nearly 30 years ago, there
is still little guidance available on their calculation. This book aims
to address this gap, through a combination of theoretical sections,
simplified examples, and real-life experiences.

This book is the result of interactions and collaborations within the
European Burden of Disease Network (COST Action CA18218), supported by
COST (cooperation in science and technology). Further information on the
network is available via \url{https://www.burden-eu.net}.

\section*{Why read this book}\label{why-read-this-book}
\addcontentsline{toc}{section}{Why read this book}

This book is primarily intended for students, researchers and public
health professionals interested in learning how to calculate DALYs.

However, it should also be noted that the \textbf{best way of learning
is by doing}. We therefore hope that this book can encourage you to get
started with your own calculation examples.

\section*{Structure of the book}\label{structure-of-the-book}
\addcontentsline{toc}{section}{Structure of the book}

The first part of the book is dedicated to the basic concepts of DALY
calculations. Starting from simple examples, different layers of
complexity will be introduced.

The second part of the book is dedicated to national burden of disease
studies.

\chapter*{About the authors}\label{about-the-authors}
\addcontentsline{toc}{chapter}{About the authors}

\section*{Brecht Devleesschauwer}\label{brecht-devleesschauwer}
\addcontentsline{toc}{section}{Brecht Devleesschauwer}

Some info here.

\url{https://twitter.com/brechtdv}

\part{Calculating DALYs}\label{part-calculating-dalys}

\chapter{Introduction}\label{introduction}

Some text.

\bibliography{book.bib,packages.bib}

\end{document}
